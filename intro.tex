\section{Introduction}
The system of Poisson and Nernst-Planck (further denoted PNP) 
equations has been widely
used to describe the charge transport --- which includes the ionic
migration, diffusion, and convection --- in a medium. 
The charge transport process is a key mechanism
for electromechanical transduction in some of the electroactive polymer
(EAP) materials, for instance, in ionic polymer-metal composites (IPMCs)~
\cite{basu1997membrane,shahinpoor2001smartmat,
nasser2002applied,newbury2003intelligent, wallmersperger2007appliedphysics,
pugal2008appliedphysics,pugal2010polymer}.
As the system of equations is nonlinear and for a domain with two
electrodes, the charge concentration differences occur in a very narrow
region near the boundaries, the required computing power for a full
scale domain is, especially in 3D, rather significant. This
requires a mesh that is optimal in terms of calculation time and calculation
error.

The Nernst-Planck equation for a mobile species ---
in this case for counter ions --- and without the convection term is:
\begin{equation}
  \frac{\partial C}{\partial t}+\nabla\cdot(-D\nabla C-z\mu FC\nabla\phi)=0,
  \label{eq:nernst-planck}
\end{equation}
where $C$ is the counter ion concentration, $D$ diffusion, $\mu$ mobility,
$F$ Faraday constant, $\phi$ voltage, and $z$ charge number. The Poisson
equation is
\begin{equation}
  -\nabla^2\phi=\frac{F\rho}{\varepsilon},
  \label{eq:poisson}
\end{equation}
where $\varepsilon$ is an absolute dielectric permittivity and the
charge density $\rho$ is expressed
\begin{equation}
  \rho=C-C_{0}
  \label{eq:rho}
\end{equation}
with $C_{0}$ being a constant anion concentration.

In the following section we give a brief overview of the motivation
of the study and a practical application of the results. 
Thereafter the weak-form~\cite{Hermes-book} derivation of the PNP system of equations
and time dependent adaptive \emph{hp}-FEM solutions of the system is presented
for different adaptivity algorithms. 
Also, advantages of a multi-mesh solution over a single-mesh solution are discussed.


