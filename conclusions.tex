\section{Conclusion and Outlook}\label{sec:conc}

In this work the system of Nernst-Planck-Poisson equations
was solved using \emph{hp}-finite element method with adaptive
multimesh configuration. The weak form, residuals and
the Jacobian matrix of the system were explicitly derived
and implemented in Hermes \emph{hp}-FEM adaptive solver.
The solution for Nernst-Planck-Poisson
problem with two field variables $C$ and $\phi$ results in 
very different field gradients in the space and time.
Conventional finite element solvers do not provide the means 
how to deal with such problems such that both the error of
the solution and problem size remain small throughout the
time dependent solving process.

It was shown that using time dependent adaptivity, multi-mesh
configuration, and anisotropic \emph{hp} refinements, the problem
size remains acceptably small throughout the solving process.
Namely, hermes refinement modes HP\_ANISO and HP\_ANISO\_P
resulted the smallest and fastest problem, respectively.
At the same time, relative error of the solution was known.
Furthermore, using the multi-mesh configuration for the variables
$C$ and $\phi$ was justified --- the adaptivity algorithm
did not refine the mesh of $\phi$ nor did increase the
polynomial degree throughout the adaptivity process. However,
the mesh was significantly refined for $C$ and also the
maximum polynomial degree was varied in the range of
$2\ldots 9$. So it is efficient to use multi-mesh in terms of
the number of degrees of freedom.

Conclusively, by using \emph{hp}-FEM with adaptive multi-mesh
configuration we can possibly reduce the problem size
of the Nernst-Planck-Poisson equation system significantly while
still maintaining prescribed precision of the solution. 
We believe, and this
is yet to be demonstrated, that this becomes especially
important when dealing with 3D problems with a large physical
domain.
