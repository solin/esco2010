\section{Conclusion and Outlook}\label{sec:conc}

In this work the system of Nernst-Planck-Poisson equations
was solved using \emph{hp}-finite element method with adaptive
multi-mesh configuration. The weak form, residuals and
the Jacobian matrix of the system were explicitly derived
and implemented in Hermes \emph{hp}-FEM time dependent
adaptive solver.
The solution for Nernst-Planck-Poisson
problem with two field variables $C$ and $\phi$ results in 
very different field gradients in the space and time.
When using a conventional low order
FEM, finding an optimal mesh for this
type of problem such that both the error of
the solution and problem size remain small throughout the
time dependent solving process is difficult. 

In the current work we showed that using the time dependent adaptivity, 
multi-mesh configuration, and anisotropic \emph{hp} refinements, the problem
size remains very small throughout the solving process while
maintaining a pre-set relative error of the solution.
Namely, Hermes refinement mode HP\_ANISO 
resulted in the smallest and fastest problem solution.
Furthermore, using the multi-mesh configuration for the physical fields
$c$ and $\varphi$ --- scaled variables for $C$ and $\phi$, respectively ---
was justified. The adaptivity algorithm
refined the meshes of $\varphi$  and $c$ and increased the
polynomial degrees of the corresponding spaces differently.
The mesh was significantly refined for $c$ and also the
maximum polynomial degree was varied in the range of
$2\ldots 9$ whereas for $\varphi$, the maximum polynomial degree
remained lower. So it is efficient to use multi-mesh in terms of
the number of degrees of freedom.

Conclusively, by using \emph{hp}-FEM with adaptive multi-mesh
configuration we can possibly reduce the problem size
of the Nernst-Planck-Poisson equation system significantly while
still maintaining prescribed precision of the solution. 
We believe, and this
is yet to be demonstrated, that this is especially
important when dealing with 3D problems in a large physical
domain with non-uniform boundary conditions.
