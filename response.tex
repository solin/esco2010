\documentclass[mathpazo]{cicp}

%\journal{Applied Mathematics and Computation}
\usepackage{graphicx}
\usepackage{amssymb}
\usepackage{amsmath}
\usepackage{amsthm}
\usepackage{amssymb}
\usepackage{amsfonts}
\usepackage{amsrefs}
%\usepackage{subfig}

\pagestyle{empty}

\begin{document}

%\title{Response to the reviewers of \emph{Modeling Ionic Polymer-Metal Composites\\ with Space-Time Adaptive Multimesh hp-FEM}}

\begin{center}
{\huge \bf Response to the Referees}
\end{center}


%\author[Deivid Pugal et.~al.]{Deivid Pugal\affil{1}\comma\affil{4},
%Pavel Solin\affil{2}\comma\affil{3}\comma\corrauth,
%Kwang J. Kim\affil{1}, and Alvo Aabloo\affil{4}}

%\address{\affilnum{1}\ Mechanical Engineering Department, University of Nevada, Reno, NV, U.S.A.\\
%\affilnum{2}\ Department of Mathematics and Statistics, University of Nevada, Reno, NV, U.S.A.\\
%\affilnum{3}\ Institute of Thermomechanics, Prague, Czech Republic\\
%\affilnum{4}\ Institute of Technology, Tartu University, Estonia}

%\emails{{\tt david.pugal@gmail.com} (D.~Pugal), {\tt solin@unr.edu} (P.~Solin),
%	{\tt kwangkim@unr.edu} (K.~Kim), {\tt alvo@ut.ee} (A.~Aabloo)}


%\begin{abstract}
%We are concerned with a model of ionic polymer-metal composite (IPMC) materials
%that consists of a coupled system of the Poisson and Nernst-Planck equations, 
%discretized by means of the finite element method (FEM). We show that due to the 
%transient character of the problem it is efficient to use adaptive algorithms 
%that are capable of changing the mesh dynamically in time. We also show 
%that due to large qualitative and quantitative differences between the 
%two solution components, it is efficient to approximate them on different 
%meshes using a novel adaptive multimesh \emph{hp}-FEM. The study is 
%accompanied with numerous computations and comparisons of the adaptive 
%multimesh \emph{hp}-FEM with several other adaptive FEM algorithms. 
%\end{abstract}

%\maketitle

\vspace{4mm}

We would like to thank the referees for their insightful comments 
that helped to improve the quality of the paper. We did our best to 
accommodate all of them. This required a substantial effort,
including the implementation of dimensionless transformation of 
the equations and re-doing all computations.


\section{Reviewer \#1}

\begin{enumerate}
\item \emph{The comparisons presented by the authors are interesting. One important aspect regarding the
Poisson-Nernst-Planck equations might be missing. A characteristic quantity of the system is the
dimensionless Debye length. It is well-known that the case of small Debye length’s provides severe
numerical difficulties. Hence, it would be very interesting to see how hp-FEM are able to handle
such cases. This might also lead to a different picture where possibly the electrostatic potential
needs more adaptivity than the concentration of counter ions in contrast to examples considered so
far.}

Section 5.6 addressing these concerns was added.

\item \emph{last sentence and Figure 1 Could the authors give a schematic picture presenting the electrostatic forces causing the deformation/bending of the IPMC?}

An explanation was added to Figure 1.

\item \emph{3rd line "materials that includes"}

Fixed.

\item \emph{2nd paragraph, The computing power ... Are there references which elucidate this statement?}

A reference was added to a previous study on solving the PNP system in 3D.

\item \emph{1st sentence, ... principal difficulties ... As mentioned above, there are also other issues
known for the Poisson-Nernst-Planck system in the case of small Debye lenghts. Are the
authors sure that the motivation given in Section is numerical more demanding? If yes, are
there references available?}

In some practical cases such as in case of modeling the Ionic Polymer Metal Composites,
often the values of diffusion constant $D$ and dielectric permittivity $\varepsilon$ are such
that the gradients of $\varphi$ do not become significantly high in a studied time scale, however,
the gradients of $c$ do. This is one reason why it is hard to find an optimal mesh, especially
in case of a 3D domain.

\item \emph{eq. (3.5) It might be more appropriate to present the dimensionless formulation of the problem instead of introducing the physically less motivated notations (3.5). Such a formulation then
immediately provides the characteristic dimensionless parameter of the system, i.e. the Debye
length.}

This was very good point. The dimensionless form of the equations was derived. All
results were re-calculated and each graph and image in the Results section was re-done.
Instead of using $C$
and $\phi$, the corresponding scaled variables $c$ and $\varphi$ are now used throughout the results section.

Furthermore, this time, all the results were calculated using adaptivity where the relative error
of the physical fields was considered separately.  Previously, absolute errors were compared
and the absolute error of the physical field $\phi$ was always very  much smaller compared to
$C$. Therefore no adaptivity of whatsoever was observed in case of $\phi$. The new results
(for instance, Figures~14 and~15) show that the mesh and the polynomial
space of physical field $\varphi$ is also adapted, however, not as much as in case of $c$.
We feel that this approach makes physically more sense.

\item \emph{3rd paragraph The reader might loose at this point the interest in reading further since there
is no motivation which favors a hp-FEM discretization over traditional low-order FEM. One
sentence regarding computational performance or a related comment might be useful~...}

A motivation sentence describing different convergence properties of low-order FEM 
and hp-FEM was added to the beginning of Section 4 along with a reference. 

\item \emph{2nd last sentence in 1st paragraph A short explanation, how Hermes further deals with this
difference without going to much into details, might be interesting here ...}

The title of this section was slightly changed (Hermes was left out), so mentioning Hermes here would not be 
appropriate. However, a better explanation was added to Paragraph 4.1 which describes Hermes. 

\item \emph{1st sentence in Sec. 5.2 Convergence plots of the error, which the authors mention, would be interesting...}

Figure 17 shows an example error convergence in case of HP\_ANISO, HP\_ANISO\_H, and HP\_ISO.

\item \emph{last paragraph "The results"}

Fixed.

\item \emph{Section 5.4, eq. (5.4) What does Hermes do in the case $e^n\geq\delta$?}

Adaptive time stepping was not supported in Hermes until mid-February 2011. Recently we have 
implemented a new method that allows Hermes to perform time integration based on arbitrary Butcher's 
tables (including adaptivity based on arbitrary embedded Runge-Kutta methods). However, another paper on this subject 
is in preparation and thus we would prefer not to discuss details in the present paper. 
The following comment was added below (5.5): \emph{At this point, the implementation does not support
adaptive time stepping if $e^n\geq \delta$. However, the implementation 
of advanced adaptive higher-order time-stepping methods is in progress.}

\end{enumerate}

\section{Reviewer \#2}

\begin{enumerate}

\item \emph{The authors have not given the initial values of $C$}

The initial value was added below Eq.~(1.1).

\item \emph{The authors have not given the size of time-step and the relation
of time step size and h; p or CFL stability condition.}

All computations performed in this paper were done using the absolutely stable 
second-order Crank-Nicolson method. This method does not require any 
limitation of time step such as the CFL condition. Of course, the 
accuracy is still an issue and the size of the time step was chosen 
based on multiple experiments, so that the size of the temporal error 
was approximately the same as the error in space. The ultimate answer 
to this issue is adaptive time stepping which is the subject of a forthcoming 
paper. We have given the size of the time step, and 
we explained in more detail how the choice of the fixed time step
for the Crank-Nicolson method was made in the very beginning
of Section 5.

\end{enumerate}
\end{document}


