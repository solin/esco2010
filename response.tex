\documentclass[mathpazo]{cicp}

%\journal{Applied Mathematics and Computation}
\usepackage{graphicx}
\usepackage{amssymb}
\usepackage{amsmath}
\usepackage{amsthm}
\usepackage{amssymb}
\usepackage{amsfonts}
\usepackage{amsrefs}
%\usepackage{subfig}

\begin{document}

\title{Response to the reviewers of \emph{Modeling Ionic Polymer-Metal Composites\\ with Space-Time Adaptive Multimesh hp-FEM}}


\author[Deivid Pugal et.~al.]{Deivid Pugal\affil{1}\comma\affil{4},
Pavel Solin\affil{2}\comma\affil{3}\comma\corrauth,
Kwang J. Kim\affil{1}, and Alvo Aabloo\affil{4}}

\address{\affilnum{1}\ Mechanical Engineering Department, University of Nevada, Reno, NV, U.S.A.\\
\affilnum{2}\ Department of Mathematics and Statistics, University of Nevada, Reno, NV, U.S.A.\\
\affilnum{3}\ Institute of Thermomechanics, Prague, Czech Republic\\
\affilnum{4}\ Institute of Technology, Tartu University, Estonia}

\emails{{\tt david.pugal@gmail.com} (D.~Pugal), {\tt solin@unr.edu} (P.~Solin),
	{\tt kwangkim@unr.edu} (K.~Kim), {\tt alvo@ut.ee} (A.~Aabloo)}


\begin{abstract}
We are concerned with a model of ionic polymer-metal composite (IPMC) materials
that consists of a coupled system of the Poisson and Nernst-Planck equations, 
discretized by means of the finite element method (FEM). We show that due to the 
transient character of the problem it is efficient to use adaptive algorithms 
that are capable of changing the mesh dynamically in time. We also show 
that due to large qualitative and quantitative differences between the 
two solution components, it is efficient to approximate them on different 
meshes using a novel adaptive multimesh \emph{hp}-FEM. The study is 
accompanied with numerous computations and comparisons of the adaptive 
multimesh \emph{hp}-FEM with several other adaptive FEM algorithms. 
\end{abstract}

\maketitle


\section{Reviewer \#1}

\begin{enumerate}
\item \emph{The comparisons presented by the authors are interesting. One important aspect regarding the
Poisson-Nernst-Planck equations might be missing. A characteristic quantity of the system is the
dimensionless Debye length. It is well-known that the case of small Debye length’s provides severe
numerical difficulties. Hence, it would be very interesting to see how hp-FEM are able to handle
such cases. This might also lead to a different picture where possibly the electrostatic potential
needs more adaptivity than the concentration of counter ions in contrast to examples considered so
far.}

TODO
\item \emph{last sentence and Figure 1 Could the authors give a schematic picture presenting the electrostatic forces causing the deformation/bending of the IPMC?}

An explanation was added to the images in Figure 1.

\item \emph{3rd line "materials that includes"}

Fixed.

\item \emph{2nd paragraph, The computing power ... Are there references which elucidate this statement?}

A reference was added to a previous study where solving the PNP system in 3D was discussed.

\item \emph{1st sentence, ... principal difficulties ... As mentioned above, there are also other issues
known for the Poisson-Nernst-Planck system in the case of small Debye lenghts. Are the
authors sure that the motivation given in Section is numerical more demanding? If yes, are
there references available?}

TODO

\item \emph{eq. (3.5) It might be more appropriate to present the dimensionless formulation of the problem instead of introducing the physically less motivated notations (3.5). Such a formulation then
immediately provides the characteristic dimensionless parameter of the system, i.e. the Debye
length.}

This was very good point. The dimensionless form of the equations was derived. $\mathbf{All}$ the
results were recalculated and each graph and image in the Results section was redone.
Instead of using $C$
and $\phi$, the corresponding scaled variables $c$ and $\varphi$ are now used throughout the results section.

Furthermore, this time, all the results were calculated using adaptivity where the relative error
of the physical fields was considered separately.  Previously, absolute errors were compared
and the absolute error of the physical field $\phi$ was always very  much smaller compared to
$C$. Therefore no adaptivity of whatsoever was observed in case of $\phi$. The new results
(for instance, Figures~14 and~15) show that the mesh and the polynomial
space of physical field $\varphi$ is also adapted, however, not as much as in case of $c$.
We feel that this approach makes physically more sense.

\item \emph{3rd paragraph The reader might loose at this point the interest in reading further since there
is no motivation which favors a hp-FEM discretization over traditional low-order FEM. One
sentence regarding computational performance or a related comment might be useful ...}

TODO Pavel

\item \emph{2nd last sentence in 1st paragraph A short explanation, how Hermes further deals with this
difference without going to much into details, might be interesting here ...}

TODO Pavel

\item \emph{1st sentence in Sec. 5.2 Convergence plots of the error, which the authors mention, would be interesting...}

Figure 13 was added to show an example error convergence in case of HP\_ANISO.

\item \emph{last paragraph "The results"}

Fixed.

\item \emph{Section 5.4, eq. (5.4) What does Hermes do in the case $e^n\geq\delta$?}

Currently Hermes does not support the time step reduction. However, this is planned to be added
in future. A comment was also added below (5.5): \emph{At this point,
the implementation does not support reducing the time step when $e^n \geq \delta$.}

\end{enumerate}

\section{Reviewer \#2}

\begin{enumerate}

\item \emph{The authors have not given the initial values of $C$}

The initial value was added below Eq.~(1.1).

\item \emph{The authors have not given the size of time-step and the relation
of time step size and h; p or CFL stability condition.}

TODO Pavel?

\end{enumerate}
\end{document}


