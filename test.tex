\documentclass[mathpazo]{cicp}

%%%%% author macros %%%%%%%%%
% place your own macros HERE
%%%%% end %%%%%%%%%

\begin{document}
%%%%% title : short title may not be used but TITLE is required.
% \title{TITLE}
% \title[short title]{TITLE}
\title{A Demonstration of the \LaTeXe \ Class File for the
Communications in Computational Physics}

%%%%% author(s) :
% single author:
% \author[name in running head]{AUTHOR\corrauth}
% [name in running head] is NOT OPTIONAL, it is a MUST.
% Use \corrauth to indicate the corresponding author.
% Use \email to provide email address of author.
% \footnote and \thanks are not used in the heading section.
% Another acknowlegments/support of grants, state in Acknowledgments section
% \section*{Acknowledgments}
%\author[T.~Lam]{T.~Lam\corrauth}
%\address{Author Address}
%\email{{\tt cicp@global-sci.com} (T.~Lam)}

% multiple authors:
% Note the use of \affil and \affilnum to link names and addresses.
% The author for correspondence is marked by \corrauth.
% use \emails to provide email addresses of authors
% e.g. below example has 3 authors, first author is also the corresponding
%      author, author 1 and 3 having the same address.
 \author[Zhang Z R et.~al.]{Zhengru Zhang\affil{1}\comma\corrauth,
       Author Chan\affil{2}, and Author Zhao\affil{1}}
 \address{\affilnum{1}\ School of Mathematical Sciences,
          Beijing Normal University,
          Beijing 100875, P.R. China. \\
           \affilnum{2}\ Department of Mathematics,
           Hong Kong Baptist University, Hong Kong SAR}
 \emails{{\tt zhang@email} (Z.~Zhang), {\tt chan@email} (A.~Chan),
          {\tt zhao@email} (A.~Zhao)}
% \footnote and \thanks are not used in the heading section.
% Another acknowlegments/support of grants, state in Acknowledgments section
% \section*{Acknowledgments}


%%%%% Begin Abstract %%%%%%%%%%%
\begin{abstract}
This paper describes the use of the \LaTeXe \ {\sf cicp.cls} class file for
setting papers for the {\it Communications in Computational Physics}.
\end{abstract}
%%%%% end %%%%%%%%%%%

%%%%% AMS/PACs/Keywords %%%%%%%%%%%
\ams{52B10, 65D18, 68U05, 68U07}
% \pacs{}
\keywords{\LaTeXe}

%%%% maketitle %%%%%
\maketitle


%%%% Start %%%%%%
\section{Introduction}
This paper is described how to use the {\sf cicp.cls}\footnote{Current verions is 2.1. Please
ensure you use the most up to date class file, available from the
global-sci homepage at http://www.global-sci.com/.} class file for publication
in the {\it Communications in Computational Physics}.
The {\sf cicp.cls} class file preserves much of the standard \LaTeXe \ interface
so that authors can easily convert their standard \LaTeXe \ {\sf article} style
files to the {\sf cicp} style.

\section{Preparation of Manuscript}
The Title Page should contain the article title, authors' names and complete affiliations,
and email addresses of all authors. The Abstract should provide a brief summary of the main findings of the paper.

\medskip

References should be cited in the text by a number in square brackets. Literature cited should appear on a separate page at the
end of the article and should be styled and punctuated using standard abbreviations for journals
(see Thomson ISI list of journal abbreviations).
 For unpublished lectures of symposia,
include title of paper, name of sponsoring society in full, and date. Give titles of unpublished reports with ''(unpublished)''
following the reference. Only articles that have been published or are in press should be included in the references.
Unpublished results or personal communications should be cited as such in the text.
Please note the sample at the end of this paper.



Equations should be typewritten whenever possible and the number placed in parentheses at the right margin.
Reference to equations should use the form ''Eq. (2.1)'' or simply ''(2.1).''
Superscripts and subscripts should be typed or handwritten clearly above and below the line, respectively.



Figures should be in a finished form suitable for publication.
Number figures consecutively with Arabic numerals.
Lettering on drawings should be of professional quality or generated by high-resolution
computer graphics and must be large enough to withstand appropriate reduction for publication.
For example, if you use {\sf MATLAB} to do figure plots,
axis labels should be at least point 18. Title should be 24 points or above. Tick marks labels
better have 14 points or above. Line width should be 2 (or above).


Illustrations in color in most cases can be accepted only if the authors defray the cost.
At the Editor's discretion a limited number of color figures each year of special interest
will be published at no cost to the author.

\section{Header Information}
The heading for any file using {\sf cicp.cls} is like this;
\begin{verbatim}

\documentclass[mathpazo]{cicp}

\begin{document}

\title{Make the Title in Title Case}

\author[An Author et.~al]{First Author\affil{1},
Second Author\affil{2}\comma\corrauth
\and Third Author\affil{1}}

\address{\affilnum{1}\ Address for first and third authors \\
\affilnum{2}\ Address for second author}

\emails{{\tt cicp@global-sci.com} (A.~Author),
{\tt second@author.email} (S.~Author),
{\tt third@author.email} (T.~Author)}

\begin{abstract}
Text here, no equation, no citation, no reference.
\end{abstract}

\ams{list here}
\pacs{list here}
\keywords{list here}

\maketitle

\section{First Section}

\end{document}

\end{verbatim}

\noindent {\bf Notes:}
\begin{enumerate}
\item Starting from volume two, we use package {\sf mathpazo}. If you do not have this package,
you just remove the option {\sf mathpazo} in $\backslash$documentclass.
We can make it for you in the printing version.
\item The first argument in square bracket of $\backslash$author is a MUST.
It is for the running heads.
$\backslash$corrauth should be provided to indicate the
corresponding author. $\backslash$email(s) is used to show that
author(s) email address(es) in footnote.
\item The abstract should be captable of standing by itself, in the
absence of the body of the article and of the bibliography. It is forced
to print within one page, so there may be problem if it is too long.
\item You may have your own macros but keep it to an absolute minimum.
\item $\backslash$thanks is not working in this style. You should use
$\backslash$section*\{Acknowlegments\} for acknowlegments/grant support as
the last section (just before references).
\end{enumerate}

\section{Some Remarks}
\subsection{Mathematics}
{\sf cicp.cls} makes the full functionality of \AmS\TeX \ available. We encourage
the use of the {\sf align}, {\sf gather} and {\sf multline} environments for displayed
mathematics.



\subsection{Cross-referencing}
The use of the \LaTeX cross-reference system for figures, tables, equations
and citations is encouraged.


%%%% Acknowledgments %%%%%%%%
\section*{Acknowledgments}
The author would like to thank  ....

%%%% Bibliography  %%%%%%%%%%
\begin{thebibliography}{99}
\bibitem{Goossens} Michel Goossens, Frank Mittelbach and Alexander Samarin,
The LaTeX Companion,
Addison-Wesley, 1994.
\bibitem{Kopka}Helmut Kopka and Patrick W.~Daly, A Guide to LaTeX,
Addison-Wesley, 1999.
\bibitem{Knuth}Donald E. Knuth, The TeXbook,
Addison-Wesley, 1992.
\bibitem{Other}A.~N.~Other, A demonstration of the LaTeX2e class file for
the International Journal for Numerical Methods in Engineering,
Int.~J.~Numer.~Meth.~Engng,  00 (2000), 1-6.
\bibitem{Yin}Z.~Yin, H.~J.~H.~Clercx and D.~C.~Montgomery,
An easily implemented task-based parallel scheme for the Fourier
pseudospectral solver applied to 2D Navier-Stokes turbulence,
Computers \& Fluids, 33 (2004), 509-520.
\end{thebibliography}

\end{document}
